\chapter{集合论} \label{chap:set}

集合是高中数学接触到的第一个概念,也是数学领域最基础的数学概念之一。在本章中,我
们将拓展一些高中内容。

\section{超集、相对补集}

我们在高中已经学习到了一些和集合相关的内容。我们知道元素和集合间的关系:
\textbf{属于};知道集合和集合间的关系:\textbf{包含于};知道集合间的运算:
\textbf{交集}、\textbf{并集}和\textbf{补集}。

我们知道,关系是相互的——小王是老王的儿子,老王是小王的父亲;那放之于集合的范畴,
我们定义了和子集相对的超集:

\begin{rawdef}[超集]\index{超集}
    集合$A$的\textbf{超集}$B$是一个集合,它含有$A$中的所有元素。记作$B \supseteq 
    A$.
\end{rawdef}

也就是说,如果$B$包含$A$,那$B$就是$A$的一个超集。这个概念是很容易理解的,因为它
并不需要什么“新知识”,只是为一种性质取了一个名字而已。

我们学过集合$A$在全集$U$中的补集,全集指的是含有“研究问题中涉及的所有元素”的集合。
但是这个指代很不清楚,我们常常搞不清楚这个“研究问题”指的是当前的一句话还是整一段
推导。但是在很多情况下,我们只是想要表示“含有所有集合$P$中的元素,且不含集合$Q$
中的元素”的集合。在这种情况下,我们定义了相对补集:

\begin{rawdef}[相对补集]\index{相对补集}
    集合$A$在集合$B$中的相对补集是由所有属于$B$但不属于$A$的元素组成的集合。记作
    $B\setminus A$或$B - A$。
\end{rawdef}

使用这种表示方法,我们便可以简单的写出一些集合。

\begin{rawexp}
    实函数$y = \dfrac{\ee^2+2x-1}{x^2-1}$的定义域是什么?
\end{rawexp}

\begin{rawsol}
    题述函数有定义的充要条件是$x^2 -1 \neq 0$,即$x \neq \pm 1$。所以题述函数的
    定义域是$\mathbb{R}\setminus \left\{ -1,1 \right\}$.
\end{rawsol}

此外,我们还可以用这种方式将无理数集表示为$\mathbb{R}\setminus\mathbb{Q}$,将纯
虚数集表示为$\mathbb{C}\setminus\mathbb{R}$。

\section{集合的基数}

我们知道,集合具有\textbf{确定性}。这一点说明,\textbf{每一个集合都有确定的元素
个数}。我们把集合的元素个数叫作这个集合的\textbf{基数}\index{基数},也叫
\textbf{势}\index{势}。对于有限集$A$,它的基数是一个自然数,记作$\card{A}
$\footnote{这里给出的是课本上的表示法,一个集合的基数的写法还有$\abs{A}$、$\# A$、
$\overline{A}$、$\overll{A}$等。}。我们直观地将集合的基数理解为集合的大小。

考虑集合$\left\{ 1,2,3 \right\} $和集合$\left\{ 2,3,4 \right\} $,它们不相等,但
是有相同的基数。对于这个例子,这个结论是显然的。但是,究竟怎么一般地比较两个集合
的元素个数?满足了什么条件,我们才能说两个集合具有相同的基数?

根据一年级教科书中的相关内容\cite{pep_math_1A},我们尝试将被比较的两个集合中的元
素\textbf{一一对应}。如果两个集合的元素可以完成这种“一一对应”,并且最后都恰好用
完了两个集合中的元素,那么我们就说,两个集合拥有相同的基数,又称\textbf{等势};如
果在其中一个集合的所有元素都用完的时候,另一个集合中还存在没有被“配对”的元素,那
么我们说,前者的基数小于后者。

\begin{rawexp}
    比较集合$A=\mathbb{Z}\cap \left[ -1,3 \right]$与集合$B=\mathbb{Z}\cap \left( 
    -1,3\right)$的大小。
\end{rawexp}

\begin{rawsol}
    对于$a \in A$,定义它在集合$B$中可能存在的“对应”$b \in B$满足$b = a$。由此我
    们得到$A$中所有在$B$中有“对应”的元素组成的集合$C$:
    \[
        C = \left\{a\,|\,a\in A,\ \exists b\in B,\ b=a\right\} = A\cap B = 
        \left\{ 0,1,2 \right\} .
    \]

    我们发现,$B\setminus C=\varnothing$,但$A\setminus C\neq\varnothing$。也就
    是说,在尝试建立一一对应关系之后,$B$中元素用完但$A$中元素有剩。所以我们说
    \[
        \card{A} > \card{B}.
    \]
\end{rawsol}

在上述的解法中,我们将“一一对应”定义为$b=a$。然而,并非要将这种“对应”定义为相等。

\begin{rawexp}[2024年台州二模T19节选]
    设$A,B$是两个非空集合,如果对于集合$A$中的任意一个元素$x$,按照某种确定的对
    应关系$f$,在集合$B$中都有唯一确定的元素$y$和它对应,并且不同的$x$对应不同的
    $y$;同时$B$中的每一个元素$y$都有一个$A$中的元素$x$与它对应,则称$f:
    A\rightarrow B$为从集合$A$到集合$B$的一一对应,并称集合$A$与$B$等势,记作
    $\overll{A}=\overll{B}$。若集合$A$与$B$之间不存在一一对应关系,则称$A$与$B$
    不等势,记作$\overll{A}\neq\overll{B}$。

    \vspace{1ex}
    求证:$\overll{\mathbb{N}^\ast} \neq \overll{\left\{ x\,|\, x\subseteq N^\ast 
    \right\}} $。
\end{rawexp}

\begin{proof}
    设集合$M=\left\{ x\,|\, x\subseteq N^\ast \right\}$,它是所有正整数集的子集
    构成的集合。

    我们尝试对$M$中除空集$\varnothing$的元素定义对应关系$f:D\rightarrow 
    \mathbb{N}^\ast$如下,其中$D=M\setminus \left\{ \varnothing \right\} $:
    \[
        f(A) = \overll{A},\quad A\subseteq \mathbb{N}^\ast.
    \]
    现在我们只需同时证明:
    \begin{enumerate}[label=(\roman*)]
        \item 正整数集的子集的基数完全覆盖了所有正整数。即
            \[
                \forall p\in \mathbb{N}^\ast,\ \exists A\subseteq \mathbb{N}
                ^\ast,\ \overll{A} = p.
            \] \label{tmp:set_enum_1}

            这句话指出,对$D$中每一元素作用映射$f$后的结果至少已经“用去”了正整数
            集中的所有元素。

        \item 以下两句的\emph{其中之一}:
            \begin{enumerate}
                \item 存在至少一对正整数集的子集,它们的基数相同。\label{tmp:set_enum_2a}
                    
                    这句话指出有存在复数个$M$中元素,它们被$f$映射的结果相同,而
                    这个结果已经被第一个元素“用去”了,即其余的几个元素是“多出来”
                    的。

                \item 存在基数不为正整数的正整数集的子集。\label{tmp:set_enum_2b}

                    这句话指出存在$M$中元素,它被$f$映射的结果不在$\mathbb{N}
                    ^\ast$中,即它是“多出来”的那个。
            \end{enumerate}
    \end{enumerate}
    \begin{proof}[对~\ref{tmp:set_enum_1}~的证明]
        $\forall p \in \mathbb{N}^\ast$,令$A = \left\{ x\in\mathbb{N}^\ast\,
        |x \le p\right\} $,则有$\overll{A}=p$。
    \end{proof}
    而~\ref{tmp:set_enum_2a}~和~\ref{tmp:set_enum_2b}~都很容易证明。我们可以轻易
    举出基数相同的子集对,如$\left\{ 1,2 \right\}$和$\left\{ 3,4 \right\}$。对于
    ~\ref{tmp:set_enum_2b},我们只需举出空集$\overll{\varnothing} \notin 
    \mathbb{N}^\ast$。

    这就证明了,我们定义的$f$不是一个一一对应。所以根据题目描述,$\overll{M}
    \neq\overll{\mathbb{N}^\ast}$
\end{proof}

\begin{exercise}\label{exe:same_card_subset}
    证明:
    \begin{enumerate}
        \item 正整数集$\mathbb{N}^\ast$和自然数集$\mathbb{N}$等势。

        \item 整数集$\mathbb{Z}$和偶数集$E = \left\{ 2k\,|\,k\in \mathbb{Z} \right\}
            $等势。

        \item 正整数集$\mathbb{N}^\ast$和$\mathbb{Z}$等势。\label{exe:pos_neg_equinum}
    \end{enumerate}
\end{exercise}

练习~\ref{exe:same_card_subset}~的结果表明,一个集合与其真子集有可能等势。这看似
是违背直觉的——我们一直认为,整体应该“大于”部分。但是由于我们正在研究无限集,这个
经验规律不是很适用。

\begin{rawthm}\label{thm:N_Q_equinum}
    自然数集$\mathbb{N}$与有理数集$\mathbb{Q}$等势。
\end{rawthm}

\begin{proof}
    由练习~\ref{exe:same_card_subset}(\ref{exe:pos_neg_equinum})~,我们可以证明
    \emph{有理数集与正有理数集等势}。我们将正有理数按照分子与分母的和从小到大排
    成一列:
    \[
        \left\{ q_n \right\} : 1,\ \frac{1}{2},\ \frac{2}{1},\ \frac{1}{3},\ \frac{3}
        {1},\ \frac{1}{4},\ \frac{2}{3},\ \frac{3}{2}\cdots
    \]
    对于任意正有理数$q_0$,总存在$i\in \mathbb{N}^\ast$使得在上述数列中$q_i=q_{0}
    $;且对于任意的$j\in\mathbb{N}^\ast$,总在上述数列中存在正有理数$q_j$。由此
    我们建立了$q_{0}\in\mathbb{Q}$与$i\in\mathbb{N}^\ast$的一个一一对应。也就是
    说,正有理数集和正整数集等势。

    因为正整数集又和自然数集等势,所以正有理数集与自然数集等势。所以我们证明了有
    理数集与自然数集等势。
\end{proof}

使用类似的方法,我们可以证明:
\begin{itemize}
    \item 拥有相同端点的开区间和闭区间等势。

    \item 所有开区间互相等势。

    \item 开区间与实数集$\mathbb{R}$等势。
\end{itemize}

这三个定理的证明略显复杂,读者可以自行查阅资料。此外,我们将演示证明下面的定理:

\begin{rawthm}
    实数集的势大于自然数集。
\end{rawthm}

