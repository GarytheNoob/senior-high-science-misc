\chapter{集合} \label{chap:set}

集合是高中数学接触到的第一个概念,也是数学领域最基础的数学概念之一。在本章中,我
们将拓展一些高中内容。本章的后半部分还使用大量篇幅介绍了集合基数相关知识,其与高
中内容联系有限,仅供读者拓展知识。

\section{超集、相对补集}

我们在高中已经学习到了一些和集合相关的内容。我们知道元素和集合间的关系:
\textbf{属于};知道集合和集合间的关系:\textbf{包含于};知道集合间的运算:
\textbf{交集}、\textbf{并集}和\textbf{补集}。

我们知道,关系是相互的——小王是老王的儿子,老王是小王的父亲;那放之于集合的范畴,
我们定义了和子集相对的概念——超集:

\begin{rawdef}[超集]\index{超集}
    集合$A$的超集$B$是一个集合,它含有$A$中的所有元素。记作$B \supseteq 
    A$.
\end{rawdef}

也就是说,如果$B$包含$A$,那$B$就是$A$的一个超集。这个概念是很容易理解的,因为它
并不需要什么“新知识”,只是为一种性质取了一个名字而已。

我们学过集合$A$在全集$U$中的补集,全集指的是含有“研究问题中涉及的所有元素”的集合。
但是这个指代很不清楚,我们常常搞不清楚这个“研究问题”指的是当前的一句话还是整一段
推导。但是在很多情况下,我们只是想要表示“含有所有集合$P$中的元素,且不含集合$Q$
中的元素”的集合。在这种情况下,我们定义了相对补集:

\begin{rawdef}[相对补集]\index{相对补集}
    集合$A$在集合$B$中的相对补集是由所有属于$B$但不属于$A$的元素组成的集合。记作
    $B\setminus A$或$B - A$。
\end{rawdef}

使用这种表示方法,我们便可以简单的写出一些集合。

\begin{rawexp}
    实函数$y = \dfrac{\ee^2+2x-1}{x^2-1}$的定义域是什么?
\end{rawexp}

\begin{rawsol}
    题述函数有定义的充要条件是$x^2 -1 \neq 0$,即$x \neq \pm 1$。所以题述函数的
    定义域是$\mathbb{R}\setminus \left\{ -1,1 \right\}$.
\end{rawsol}

此外,我们还可以用这种方式将无理数集表示为$\mathbb{R}\setminus\mathbb{Q}$,将虚
数集表示为$\mathbb{C}\setminus\mathbb{R}$。

我们高中所学习的$\complement_U A$的写法也可以写作$A^\complement$。在第二种情况下,
全集$U$需依照上下文确定。

有关补集的一个重要定律是所谓的反演律,它揭示了交集、并集、补集三者间的某种关系。

\begin{rawthm}[反演律]\label{thm:de-morgan-set}
    在全集中,若干个集合并集的补集是每一个集合补集的交集;若干个集合交集的补集是
    每一个集合补集的并集。

    对于两个集合的情况,在全集中有
    \begin{align*}
        \left( A \cup B \right)^\complement=A^\complement \cap B^\complement,
        \tag{$\ast$}\label{eq:de_morgan_cup}\\
        \left( A \cap B \right)^\complement=A^\complement \cup B^\complement.
    \end{align*}
\end{rawthm}

图~\ref{fig:venn_notAorB},~\ref{fig:venn_notA},~\ref{fig:venn_notB}~给出了%
~(\ref{eq:de_morgan_cup})~式涉及的几个集合的韦恩图。可以看得出来,当我们取图%
~\ref{fig:venn_notA}~和图~\ref{fig:venn_notB}~的公共部分——即$A^\complement \cap 
B^\complement$——刚好就可以得到图~\ref{fig:venn_notAorB}~所示区域。

\begin{figure}[H]
    \centering
    \begin{venndiagram2sets}[radius=0.8cm]
        \fillNotAorB
    \end{venndiagram2sets}
    \caption{$\left( A\cup B \right)^\complement $的韦恩图}
    \label{fig:venn_notAorB}
\end{figure}

\begin{figure}[H]
    \centering
    \begin{minipage}{4cm}
        \centering
        \begin{venndiagram2sets}[radius=0.8cm]
            \fillNotA
        \end{venndiagram2sets}
        \caption{$A^\complement$的韦恩图}
        \label{fig:venn_notA}
    \end{minipage}
    \quad
    \begin{minipage}{4cm}
        \centering
        \begin{venndiagram2sets}[radius=0.8cm]
            \fillNotB
        \end{venndiagram2sets}
        \caption{$B^\complement $的韦恩图}
        \label{fig:venn_notB}
    \end{minipage}
\end{figure}

根据定理~\ref{thm:de-morgan-set}以及其他的集合运算的性质,我们可以推导出:任意集
合之间的恒成立式,如果其在形式上只由代表集合的字母、全集$U$、空集$\varnothing$及
以下符号组成
\[
    \subseteq,\ \supseteq,\ \cup,\ \cap,\ \complement\,\text{(补集)},
\]
那么在形式上将$U$和$\varnothing$互换、$\subseteq$和$\supseteq$互换、$\cup$和
$\cap$互换,仍能得到恒成立式我们这两个恒成立式叫作\textbf{对偶的}\index{对偶}。
也就是说:\emph{集合的恒成立式总是成对出现}。这个规律适用于所有集合的恒成立式。

比如,$A \cap B \subseteq A$的对偶恒成立式是$A \cup B \supseteq A$;$A \cap 
\varnothing = \varnothing$的对偶是$A \cup U = U$。 


\section{笛卡儿积}

\subsection{二元笛卡儿积}

一副扑克牌,去掉大小王之后有52张。它们包括黑桃$\spadesuit$、红心$\heartsuit$、
草花$\clubsuit$和方块$\diamondsuit$四个花色,每个花色有A--K\footnote{A,2,3,4,5,%
6,7,8,9,10,J,Q,K}共13张。

如何用集合方便地表述一副去掉大小王的扑克牌呢?我们自然可以使用列举法:
\begin{align*}
    \{\, &\spadesuit \ace,\ \spadesuit 2,\ \spadesuit 3,\cdots,\ \spadesuit \queen,\ 
    \spadesuit \king, \\
       &\heartsuit \ace,\ \heartsuit 2,\ \heartsuit 3,\cdots,\ \heartsuit \queen,\ 
    \heartsuit \king, \\
       &\clubsuit \ace,\ \clubsuit 2,\ \clubsuit 3,\cdots,\ \clubsuit \queen,\ 
    \clubsuit \king, \\
       &\diamondsuit \ace,\ \diamondsuit 2,\ \diamondsuit 3,\cdots,\ \diamondsuit \queen,\ 
    \diamondsuit \king\, \}.
\end{align*}
列举法的缺点是显而易见的:太过冗长,并且不宜直观体现这一集合的性质。我们可以使用
另一种方式。我们先列出花色集合和点数集合:
\begin{align*}
    S &= \left\{ \spadesuit,\heartsuit,\clubsuit,\diamondsuit \right\},\\
    R &= \left\{ \ace,2,3,4,5,6,7,8,9,10,\jack,\queen,\king \right\}.
\end{align*}
这52张牌的一张可以以这种方式确定:先从$S$中选一个花色,比如$\clubsuit$;再从$R$
中选一个花色,比如4。这样我们便确定了$\clubsuit 4$这张牌,在这里我们记作$\left( 
\clubsuit, 4\right) $。这是一个\textbf{有序对}\index{有序对}。

这样一来,52张牌组成的集合就可以表示为:
\[
    D = \left\{ \left( s,r \right) \,|\, s\in S,\, r\in R \right\}.
\]

由此,我们做出如下定义。

\begin{rawdef}[笛卡儿积]\index{笛卡儿积}
    两个集合$A$和$B$的笛卡儿积,又称直积\index{直积|see{笛卡儿积}},是所有满足的
    有序对组成的集合,有序对的第一个对象是$A$中元素,第二个对象是$B$中元素,记作
    \[
        A \times B = \left\{ \left( a,b \right) \,|\, a\in A,\, b\in B \right\}.
    \]
\end{rawdef}

这样一来,上面扑克牌中的$D$也可以表示为$S\times R$。

请注意,虽然使用了代数乘法的符号,笛卡儿积运算并不满足交换律。也就是说,当$A\neq
B$时,$A \times B \neq B \times A$。其原因在于有序对。顾名思义,有序对是\emph{有
序}的——$\left( a,b \right)$与$\left( b,a \right) $表示的并非相同的东西,除非
$a=b$。

根据定义,当参与笛卡儿积运算的两个集合有一个是空集的时候,其结果也是空集,因为我
们无法从空集中取出元素构成有序对:$A \times \varnothing = \varnothing \times A = 
\varnothing$。

此外,笛卡儿积对集合的并和交满足分配率,即
\begin{align*}
    A\times (B\cup C)=(A\times B)\cup (A\times C),\\
    (B\cup C)\times A=(B\times A)\cup (C\times A),\\
    A\times (B\cap C)=(A\times B)\cap (A\times C),\\
    (B\cap C)\times A=(B\times A)\cap (C\times A),\\
    (A\times B)\cap (C\times D)=(A\cap C)\times (B\cap D).
\end{align*}


笛卡儿积很适合用来表示“所有可能的结果”,如例~\ref{exp:dice}。

\begin{rawexp}\label{exp:dice}
    投掷一个六面体骰子两次,用集合表达所有可能的结果。
\end{rawexp}

\begin{rawsol}
    容易知道投掷一次的结果
    \[
        S = \left\{ 1,2,3,4,5,6 \right\}.
    \]

    对于投掷两次的情形,所有的结果为
    \begin{align*}
        R &= S \times S\\
          &= \left\{ \left( s_{1},s_{2} \right) \,|\, s_{1}\in S,\,s_{2}\in S \right\} .
    \end{align*}
\end{rawsol}

按照这种方法,我们可以将二维坐标平面上的点的集合记作$\mathbb{R}\times\mathbb{R}$,
甚至$\mathbb{R}^2$。

\subsection{多元笛卡儿积}

有了二元笛卡儿积的概念,我们可以将其推广到多元。

\begin{rawdef}[$n$-元笛卡儿积]\index{n元笛卡儿积@$n$-元笛卡儿积}
    对于正整数$n$,$n$-元笛卡儿积定义为:对于$n$个集合$X_{1},X_{2},\cdots,X_{n}$,
    \begin{align*}
        \prod_{i=1}^n X_{i}&\coloneq X_{1}\times X_{2}\times \cdots \times X_{n} \\
                           &\coloneq \left\{ \left( x_{1},x_{2},\cdots,x_{n} 
                               \right) \,|\, x_{1} \in X_{1},\, x_{2}\in X_{2},
                           \cdots,\,x_{n}\in X_{n}\right\}.
    \end{align*}
\end{rawdef}

符号$\coloneq$表示“被定义为”,指的是前面的被定义等于后面的。它也可以反着写作
$\eqcolon$,表示“定义为”。\index{$\coloneq$}

可见,$n$-元笛卡儿积中的每一个元素都是一个\textbf{$n$-元组}\index{n元组@$n$-元
组},它相当于有$n$个对象(而不再是两个)的“有序对”,不过叫另一个名字。

和二元笛卡儿积类似,$n$-元笛卡儿积同样被用来列举所有可能的元素。比如,一天有24小
时,每小时有60分钟,每分钟有60秒。那么我们列出一天中小时、分钟和秒钟的集合:
\begin{align*}
    H &= \left\{ 0,1,2,\cdots,22,23 \right\},\\
    M &= \left\{ 00,01,02,\cdots,58,59 \right\},\\
    S &= \left\{ 00,01,02,\cdots,58,59 \right\}.
\end{align*}
这样以来,一天中的每一秒钟可以表示为$H\times M\times S$。

和二维的情况类似,三维坐标空间也常被表示为$\mathbb{R}\times\mathbb{R}
\times\mathbb{R}$,或$\mathbb{R}^3$。

\section{映射的概念}

我们已经学习过函数,知道它的三个要素:定义域、值域和对应法则。我们知道,一个函数
总将定义域中的一个元素对应到值域中的一个确定的元素。然而,在中学范围内我们研究的
都是数集上的函数。在这一节中,我们将介绍抽象化的“映射”概念。在数学中,“映射”和
“函数”表达的实质是相同的。相比来说,我们高中所学的函数着重强调其算数上的性质——单
调性、奇偶性、极大(小)值等,但本节介绍的则是纯粹的在“对应关系”层面上的性质。

\subsection{定义域、陪域和值域}

\begin{rawdef}[映射]\index{映射}
    设$A$和$B$是两个非空集合,若对$A$中任意的元素$x$,依照某总确定的规律或法则
    $f$,总有$B$中唯一确定的元素$y$与之对应,则称此对应规律或法则$f$为一个从$A$
    到$B$的映射。记作
    \[
        f: A \to B.
    \]
    如果$f$将$A$中元素$x$映射到$B$中元素$y$,则记作$f: A\to B: x\mapsto y$。

    用形式语言表述,一个对应关系$f$是$A$到$B$的映射的一个充要条件是
    \[
        \forall x \in A, \exists y \in B, f(x)=y.
    \]
\end{rawdef}

我们称$A$为映射$f: A \to B$的\textbf{定义域}\index{定义域};称$B$为其\textbf{陪
域}\index{陪域},又叫\textbf{到达域}\index{到达域|see{陪域}};称$A$中所有元素被
$f$作用的结果组成的集合$\left\{ f(x) \,|\, x\in A \right\} $叫作\textbf{值域}
\index{值域}。有时候也把$\left\{ f(x) \,|\, x\in A \right\} $简记作$f(A)$。

请注意,陪域和值域是不同概念。陪域总是值域的超集。记号$f: A \to B$只是说明了$f$
对于$A$中元素的作用结果总是$B$中的元素,并没有说明所有$B$中的元素都会被使用。

\subsection{单射、满射和双射}

一个映射的陪域不一定是它的值域。对于那些值域和陪域相等的映射,我们叫它\textbf{满
射}。

\begin{rawdef}[满射]\index{满射}
    $f:X\to Y$是满射,当其陪域与值域相等。即对于陪域$Y$中的任意元素$y$,在定义域
    $X$中存在一个元素$x$使得$f(x)=y$。

    用形式语言表述,$f:X\to Y$是满射的一个充要条件是
    \[
        \forall y \in Y, \exists x \in X, f(x)=y.
    \]
\end{rawdef}

图~\ref{fig:map_onto}~表达了一个满射。从中我们可以知道,陪域中的每一个元素都是某
一个定义域中元素的作用结果。在图上的体现是,陪域中的每一个元素都被至少被一个箭头
指着。所以“满”射指的是一个映射作用的结果“填满了”整个陪域。

\begin{figure}[ht]
\centering
\begin{minipage}{12em}
    \centering
    \begin{tikzpicture}[scale=.5]
        \draw[thick] (0,0) circle [x radius=1.5, y radius=3.8]
            (4,0) circle [x radius=1.5, y radius=3];
        \node at (0,-5) {定义域};
        \node at (4,-5) {陪域};  

        \node (a1) at (0,2.4)   {1};
        \node (a2) at (0,0.8)   {2};
        \node (a3) at (0,-0.8)  {3};
        \node (a4) at (0,-2.4)  {4};

        \node[circle] (b1) at (4,2)  {$a$}; 
        \node[circle] (b2) at (4,0)  {$b$};
        \node[circle] (b3) at (4,-2) {$c$};

        \draw[thick,->] (a1.east) -- (b1);
        \draw[thick,->] (a2.east) -- (b1);
        \draw[thick,->] (a3.east) -- (b2);
        \draw[thick,->] (a4.east) -- (b3);
    \end{tikzpicture}    
    \caption{一个满射的示意}\label{fig:map_onto}
\end{minipage}
\qquad
\begin{minipage}{12em}
    \centering
    \begin{tikzpicture}[scale=.5]
        \draw[thick] (0,0) circle [x radius=1.5, y radius=3.8]
            (4,0) circle [x radius=1.5, y radius=4.5];
        \node at (0,-5.5) {定义域};
        \node at (4,-5.5) {陪域};  

        \node (a1) at (0,2.4)   {1};
        \node (a2) at (0,0.8)   {2};
        \node (a3) at (0,-0.8)  {3};
        \node (a4) at (0,-2.4)  {4};

        \node[circle] (b1) at (4,4)  {$a$}; 
        \node[circle] (b2) at (4,2)  {$b$}; 
        \node[circle] (b3) at (4,0)  {$c$};
        \node[circle] (b4) at (4,-2) {$d$};
        \node[circle] (b5) at (4,-4) {$e$};

        \draw[thick,->] (a1.east) -- (b5);
        \draw[thick,->] (a2.east) -- (b1);
        \draw[thick,->] (a3.east) -- (b2);
        \draw[thick,->] (a4.east) -- (b4);
    \end{tikzpicture}    
    \caption{一个单射的示意}\label{fig:map_one2one}
\end{minipage}
\end{figure}

你会发现,\emph{研究映射时,规定定义域和陪域是很重要的。}在高中学习函数时,我们
只学到定义域和对应法则是很重要的。但是在研究映射的时候,规定好陪域同样重要,因为
这涉及到这个映射是否为满射。

比如,考虑两个映射:
\[
    \begin{aligned}
        f: \mathbb{R} &\to \mathbb{R}\\
        x &\mapsto x^2
    \end{aligned}
    \quad \text{和}\quad
    \begin{aligned}
        g: \mathbb{R} &\to \left[ 0,+\infty \right)\\
        x &\mapsto x^2.
    \end{aligned}
\]
它们的定义域和对应法则完全相同——都是把实数$x$映射到$x^2$。然而我们说,$f$不是满
射,而$g$是满射,因此$f$和$g$是两个不同的映射。

\begin{rawdef}[单射]\index{单射}
    映射$f: X\to Y$是单射,当它将不同的输入值对应到不同的函数值上。

    用形式语言表述,$f: X\to Y$是单射的一个充要条件是
    \[
        \forall x_{1},x_{2} \in X, f\left(x_{1}\right)=f\left(x_{2}\right) 
        \Leftrightarrow x_{1}=x_{2}
        .
    \]
\end{rawdef}

图~\ref{fig:map_one2one}~表达了一个单射。我们可以由此知道:
\begin{rawthm}\label{thm:monotone_one2one}
    单调函数都是单射。
\end{rawthm}

如果有一个映射,它既是单射、又是满射,那么我们称它为\textbf{双射}。

\begin{rawdef}[双射]\index{双射}
    映射$f:X\to Y$是双射,当它既是满射,又是单射。在这种情况下,对于定义域$X$内
    每一元素$x$,存在唯一一个$Y$内的元素$y$与之对应;对于陪域$Y$内的每一元素$y$,
    存在唯一一个$X$内的元素$x$与之对应。记作$f: X\leftrightarrow Y$。

    特殊地,当$Y=X$时,双射$f: X\leftrightarrow X$也叫作\textbf{置换}
    \index{置换|see{双射}}\index{置换}

    用形式语言表述,$f: X\to Y$是双射的一个充要条件是
    \[
        \left\{
        \begin{matrix}
            \forall y \in Y, \exists x \in X, f(x)=y.\\
            \forall x_{1},x_{2}\in X, x_{1}\neq x_{2}\Rightarrow f(x_{1})\neq f(x_{2}).
        \end{matrix}
        \right.
    \]
\end{rawdef}

\begin{figure}[ht]
    \centering
    \begin{tikzpicture}[scale=.5]
        \draw[thick] (0,0) circle [x radius=1.5, y radius=3.8]
            (4,0) circle [x radius=1.5, y radius=3.8];
        \node at (0,-5.5) {定义域};
        \node at (4,-5.5) {陪域};  

        \node (a1) at (0,2.4)   {1};
        \node (a2) at (0,0.8)   {2};
        \node (a3) at (0,-0.8)  {3};
        \node (a4) at (0,-2.4)  {4};

        \node[circle] (b1) at (4,2.4)  {$a$}; 
        \node[circle] (b2) at (4,0.8)  {$b$}; 
        \node[circle] (b3) at (4,-0.8)  {$c$};
        \node[circle] (b4) at (4,-2.4) {$d$};

        \draw[thick,->] (a1.east) -- (b3);
        \draw[thick,->] (a2.east) -- (b1);
        \draw[thick,->] (a3.east) -- (b2);
        \draw[thick,->] (a4.east) -- (b4);
    \end{tikzpicture}    
    \caption{一个双射的示意}\label{fig:map_biject}
\end{figure}

图~\ref{fig:map_biject}~表达了一个双射。根据定理~\ref{thm:monotone_one2one},我
们可以推出:
\begin{rawthm}\label{thm:monotone_biject}
    如果陪域为值域,则单调函数都为双射。
\end{rawthm}

\section{集合的基数}

人类先在生活中认识到了各种各样的集合,才逐渐形成了数(自然数)的概念。所以说,自
然数是用来数集合中元素的个数的。人类在自然界见到了各种各样的集合,有一些集合看起
来不同,但是我们的祖先冥冥之中发掘了其中的某种联系:比如,鸟的(两只)翅膀和人的
(两只)眼睛,它们显然不同,但是都可以表达数“二”。这种“拥有相同元素个数”的集合越
来越多,人们便从中总结出了数的概念。\cite{dantzig2007number}

在数学中,我们用\textbf{基数}\index{基数}或\textbf{势}\index{势|see{基数}}来表示
集合中元素的“个数”。但是,和日常语言中的“个数”不同的是,势的概念不受限于有限集的
情形。有限集的基数的意义和日常用语的“个数”相同,比如,集合$\left\{ 3,5,7 \right\}
$的基数是3。无限集基数的意义则在于比较集合的大小。

集合$A$的基数有很多种表示方法,课本上使用了$\card A$,其中card是英文cardinal(基
数)的缩写。此外,还有$\abs{A}$、$\abs{\abs{A}}$、$\overll{A}$等。
\index{$\card A$}\index{$\card A$|see{基数}}

\subsection{有限集的基数、幂集}

对于有限集,它的基数就是它含有的元素个数。我们知道,整体大于部分,所以下面的定理
是成立的。

\begin{rawthm}\label{thm:finiteset_card_subset}
    一个有限集$A$的子集$B$的基数不大于$A$的基数。即当$A$是有限集,
    \[
        B \subseteq A \Rightarrow \card B \le \card A.
    \]
\end{rawthm}

比如,由于$P = \left\{ 1,4,6,7 \right\} $是$Q = \left\{ 1,2,3,4,5,6,7 \right\} $
的子集,所以$\card P \le \card Q$。定理~\ref{thm:finiteset_card_subset}~将会在第
XX章证明——在我们介绍了数学归纳法之后。% TODO: chap

我们知道一个$n$-元有限集\index{n元有限集@$n$-元有限集}(基数为$n$的有限集)的子
集个数是$2^n$。这很容易理解:在构建一个子集的时候,原集合中的每一个元素都有两种
选择——选入和不选入。所以对于原集合中的$n$个元素,就总共有$2^n$种选择方式。

\begin{rawdef}[幂集]\index{幂集}
    \index{$\mathcal{P}(A)$}\index{$\mathcal{P}(A)$|see{幂集}}
    一个集合$A$的幂集是所有$A$的子集组成的集合,记作
    \[
        \mathcal{P}(A) = \left\{ a\,|\, a\subseteq A \right\} .
    \]
\end{rawdef}

所以上面的结论可以表述成下面的定理。

\begin{rawthm}
    $n$-元有限集$A$的幂集$\mathcal{P}(A)$的基数为$2^n$。即
    \[
        \card \left( \mathcal{P}(A) \right) = 2^{\card A}.
    \]
\end{rawthm}

\subsection{用映射比较集合基数大小}\index{映射}

我们在前一节了解了映射,知道了它是在两个集合之内的一个对应关系。通过映射,我们可
以有效比较集合的基数大小。

对于两个集合$A$和$B$,我们可以尝试构造一个映射$f: A \to B$。根据这个映射的性质,
我们可以区分以下几种情况。

\begin{description}
    \item[单射]\index{单射} 如果$f$是一个单射,那么$A$的基数不大于$B$的基数,即
        $\card A \le \card B$。

    \item[满射]\index{满射} 如果$f$是一个满射,那么$A$的基数不小于$B$的基数,即
        $\card A \ge \card B$。

    \item[双射]\index{双射} 如果$f$既是单射也是满射,即$f$是双射,那么$A$与$B$的
        基数相等,即$A$和$B$\textbf{等势}\index{等势}。
\end{description}

