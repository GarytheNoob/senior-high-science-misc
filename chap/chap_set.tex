\chapter{集合论} \label{chap:set}

集合是高中数学接触到的第一个概念,也是数学领域最基础的数学概念之一。在本章中,我
们将拓展一些高中内容。

\section{超集、相对补集}

我们在高中已经学习到了一些和集合相关的内容。我们知道元素和集合间的关系:
\textbf{属于};知道集合和集合间的关系:\textbf{包含于};知道集合间的运算:
\textbf{交集}、\textbf{并集}和\textbf{补集}。

我们知道,关系是相互的——小王是老王的儿子,老王是小王的父亲;那放之于集合的范畴,
我们定义了和子集相对的超集:

\begin{rawdef}[超集]\index{超集}
    集合$A$的\textbf{超集}$B$是一个集合,它含有$A$中的所有元素。记作$B \supseteq 
    A$.
\end{rawdef}

也就是说,如果$B$包含$A$,那$B$就是$A$的一个超集。这个概念是很容易理解的,因为它
并不需要什么“新知识”,只是为一种性质取了一个名字而已。

我们学过集合$A$在全集$U$中的补集,全集指的是含有“研究问题中涉及的所有元素”的集合。
但是这个指代很不清楚,我们常常搞不清楚这个“研究问题”指的是当前的一句话还是整一段
推导。但是在很多情况下,我们只是想要表示“含有所有集合$P$中的元素,且不含集合$Q$
中的元素”的集合。在这种情况下,我们定义了相对补集:

\begin{rawdef}[相对补集]\index{相对补集}
    集合$A$在集合$B$中的相对补集是由所有属于$B$但不属于$A$的元素组成的集合。记作
    $B\setminus A$或$B - A$。
\end{rawdef}

使用这种表示方法,我们便可以简单的写出一些集合。

\begin{rawexp}
    实函数$y = \dfrac{\ee^2+2x-1}{x^2-1}$的定义域是什么?
\end{rawexp}

\begin{rawsol}
    题述函数有定义的充要条件是$x^2 -1 \neq 0$,即$x \neq \pm 1$。所以题述函数的
    定义域是$\mathbb{R}\setminus \left\{ -1,1 \right\}$.
\end{rawsol}

此外,我们还可以用这种方式将无理数集表示为$\mathbb{R}\setminus\mathbb{Q}$,将纯
虚数集表示为$\mathbb{C}\setminus\mathbb{R}$。



