\chapter{集合论} \label{chap:set}

集合是高中数学接触到的第一个概念,也是数学领域最基础的数学概念之一。在本章中,我
们将拓展一些高中内容。

\section{超集、相对补集}

我们在高中已经学习到了一些和集合相关的内容。我们知道元素和集合间的关系:
\textbf{属于};知道集合和集合间的关系:\textbf{包含于};知道集合间的运算:
\textbf{交集}、\textbf{并集}和\textbf{补集}。

我们知道,关系是相互的——小王是老王的儿子,老王是小王的父亲;那放之于集合的范畴,
我们定义了和子集相对的超集:

\begin{rawdef}[超集]\index{超集}
    集合$A$的\textbf{超集}$B$是一个集合,它含有$A$中的所有元素。记作$B \supseteq 
    A$.
\end{rawdef}

也就是说,如果$B$包含$A$,那$B$就是$A$的一个超集。这个概念是很容易理解的,因为它
并不需要什么“新知识”,只是为一种性质取了一个名字而已。

我们学过集合$A$在全集$U$中的补集,全集指的是含有“研究问题中涉及的所有元素”的集合。
但是这个指代很不清楚,我们常常搞不清楚这个“研究问题”指的是当前的一句话还是整一段
推导。但是在很多情况下,我们只是想要表示“含有所有集合$P$中的元素,且不含集合$Q$
中的元素”的集合。在这种情况下,我们定义了相对补集:

\begin{rawdef}[相对补集]\index{相对补集}
    集合$A$在集合$B$中的相对补集是由所有属于$B$但不属于$A$的元素组成的集合。记作
    $B\setminus A$或$B - A$。
\end{rawdef}

使用这种表示方法,我们便可以简单的写出一些集合。

\begin{rawexp}
    实函数$y = \dfrac{\ee^2+2x-1}{x^2-1}$的定义域是什么?
\end{rawexp}

\begin{rawsol}
    题述函数有定义的充要条件是$x^2 -1 \neq 0$,即$x \neq \pm 1$。所以题述函数的
    定义域是$\mathbb{R}\setminus \left\{ -1,1 \right\}$.
\end{rawsol}

此外,我们还可以用这种方式将无理数集表示为$\mathbb{R}\setminus\mathbb{Q}$,将纯
虚数集表示为$\mathbb{C}\setminus\mathbb{R}$。

\section{集合的基数}

我们知道,集合具有\textbf{确定性}。这一点说明,\textbf{每一个集合都有确定的元素
个数}。我们把集合的元素个数叫作这个集合的\textbf{基数}\index{基数},也叫
\textbf{势}\index{势}。对于有限集$A$,它的基数是一个自然数,记作$\card{A}
$\footnote{这里给出的是课本上的表示法,一个集合的基数的写法还有$\abs{A}$、$\# A$、
$\overline{A}$、$\overll{A}$等。}。我们直观地将集合的基数理解为集合的大小。

考虑集合$\left\{ 1,2,3 \right\} $和集合$\left\{ 2,3,4 \right\} $,它们不相等,但
是有相同的基数。对于这个例子,这个结论是显然的。但是,究竟怎么一般地比较两个集合
的元素个数?满足了什么条件,我们才能说两个集合具有相同的基数?

根据一年级教科书中的相关内容\cite{pep_math_1A},我们尝试将被比较的两个集合中的元
素\textbf{一一对应}。如果两个集合的元素可以完成这种“一一对应”,并且最后都恰好用
完了两个集合中的元素,那么我们就说,两个集合拥有相同的基数,又称\textbf{等势};如
果在其中一个集合的所有元素都用完的时候,另一个集合中还存在没有被“配对”的元素,那
么我们说,前者的基数小于后者。

\subsection{有限集的基数}

对于有限集,这是很容易理解的。托比阿斯·丹齐格(Tobias Dantzig)曾在他的文章中举
过一个例子。\cite{dantzig2007number}

\begin{quotation}
    ……然而,尽管看起来很奇怪,我们可以在不借助计数技巧的情况下,得出一个逻辑上明
    确的数字概念。

    我们进入一个大厅。在我们面前有两组东西:观众席和观众。我们无需计数即可确定这
    两组是否相等,如果不相等,哪一组更大。因为如果每个座位都有人坐满,没有人站着,
    我们无需计数就知道这两组是相等的。如果每个座位都有人坐满,观众中有些人站着,
    我们无需计数就知道人数比座位多。
\end{quotation}

在这里,丹齐格所说的“座位上都有人”就是一种座位集合和人集合间的一一对应。如果反过
来,每个人都已有座位坐下,还有空座位,那么我们就知道座位数比人数多。丹齐格在这里
说“无须计数”,是因为我们先认识了这种“一一对应”的法则,才逐渐有了数的认识。丹齐格
在文章里还说:

\begin{quotation}
    乍一看,对应过程似乎只能用于比较两组集合,但无法在绝对意义上创建数字概念。然
    而,从相对数字到绝对数字的过渡并不困难。只需要创建模型集合,每个模型集合代表
    一种可能的集合。然后,估计任何给定集合就简化为在可用模型中选择一个可以与给定
    集合逐个对应的模型。

    原始人类在其直接环境中找到了这样的模型:鸟的翅膀可以象征数字二,三叶草象征数字三,
    动物的腿象征数字四,他自己手上的手指象征数字五。许多原始语言中可以找到这种数字词
    起源的证据。当然,一旦这种方法被采用,数字概念就得到了发展和普及。
\end{quotation}

按照丹齐格的说法,我们通过数数来比较有限集$A$和$B$的过程实质上是把$A$中元素和$B$
中元素都对应到自然数集,再比较的过程。

\subsection{无限集的基数}

而对于“没有终结”的无限集,情况则变得复杂。也许你会认为所有无限集的基数相等,都为
“无限”或“正无穷”,但事实上,“无限”之间也可以比大小。

\subsubsection{希尔伯特旅馆}

一个有趣的例子是所谓的“希尔伯特旅馆”。假设有一家拥有无限个房间的旅馆,每
个房间都有自己唯一的正整数编号:$1,2,3,4,\ldots $。某一天,这个旅馆\emph{每个房间
都客满了}。也许你好奇这样的一间旅馆是怎么住满的,但我们无须太在意这个。只需记住
在这一天,每一个正整数房间号所对应的房间里都是有客人的。

这时候,来了一个新客人想入住旅馆。我们可能认为,既然旅馆已经住满,那想必没办法再
安排这位客人入住了。但是这时酒店的经营者想出了一个办法:\emph{让每一个房间中的客
人移动到下一个房间里去}。也就是说:1号房间的客人移到2号房间,2号房间的客人移动到
3号房间,……,$n$号房间的客人移动到$n+1$号房间。这样一来,我们便空出了1号房间,
刚好够容纳这位新客人的了。

这位客人刚入住完毕,突然又来了10组客人,也想办理入住。这次,酒店的经营者使用了类
似的办法,让$n$号房间的客人移动到$n+10$号房间,成功空出了1--10号房间,安置了10组
新客人。

一会儿之后(旅馆仍然保持“客满”的状态),来了\emph{无限}多个客人。这些客人同属一
个旅游团,每人都有自己的正整数编号,并希望入住旅馆,每人得到一个新房间。这下,经
营者不能使用先前“后移”的方法了。但是,聪明的经营者想到了另一个方法:\emph{让每一
个房间中的客人移动到自己房间号的两倍对应的房间里去}。也就是说:1号房的客人到2号
房,2号房的客人到4号房,3号房的客人到6号房,……,$n$号房的客人到$2n$号房。原来旅
馆中的客人们经过这次变换,都住进了房号为偶数的房间——$2,4,6,8,\ldots $;而奇数房
——$1,3,5,7,\ldots $则都空了出来。这些奇数房恰好可以用来安置我们新来的无限位客人,
只需让编号为$k$的新客人入住$2k-1$号房就好了。

再后来(旅馆仍然客满),来了\emph{无限}多个旅游团,每个旅游团都有\emph{无限}多个
客人。这些旅游团每个都有各自的正整数编号,且每一个旅游团内的客人都类似上一段中所
讲的那样。而经营者同样有办法招待这些客人们。首先,他像上一段中那样空出了所有的奇
数房间,再让第1个旅游团的第$n$个客人入住$3^n$号房间,第2个旅游团的第$n$个客人入
住$5^n$号房间,以此类推,第$i$个旅游团的客人入住第$p_{i+1}^n$号房间(其中$p_{i+1}
$是第$i+1$个质数)。这下,每个新来的客人都有房间住了。

